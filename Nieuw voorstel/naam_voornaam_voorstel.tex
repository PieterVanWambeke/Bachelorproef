%==============================================================================
% Sjabloon onderzoeksvoorstel bachelorproef
%==============================================================================
% Gebaseerd op LaTeX-sjabloon ‘Stylish Article’ (zie voorstel.cls)
% Auteur: Jens Buysse, Bert Van Vreckem

\documentclass[fleqn,10pt]{voorstel}

%------------------------------------------------------------------------------
% Metadata over het voorstel
%------------------------------------------------------------------------------

\JournalInfo{HoGent Bedrijf en Organisatie}
\Archive{Bachelorproef 2017 - 2018} % Of: Onderzoekstechnieken

%---------- Titel & auteur ----------------------------------------------------

% TODO: geef werktitel van je eigen voorstel op
\PaperTitle{De toekomst van server based computing}
\PaperType{Onderzoeksvoorstel Bachelorproef} % Type document

% TODO: vul je eigen naam in als auteur, geef ook je emailadres mee!
\Authors{Pieter Van Wambeke\textsuperscript{1}} % Authors
\CoPromotor{Stephan Teerlinck\textsuperscript{2} (Orbid)}
\affiliation{\textbf{Contact:}
  \textsuperscript{1} \href{mailto:pieter.vanwambeke.u8838@student.hogent.be}{pieter.vanwambeke.u8838@student.hogent.be};
  \textsuperscript{2} \href{mailto:stephan.teerlinck@orbid.be}{stephan.teerlinck@orbid.be};

}

%---------- Abstract ----------------------------------------------------------

\Abstract{In dit onderzoek zal worden nagegaan of het einde in zicht is van server based computing in de nabije toekomst, meer precies of het einde in zicht is van Citrix en Microsoft RDS. Dit werk is belangrijk omdat meer en meer bedrijven overschakelen naar alternatieven in plaats van SBC. Maar hoe vindt dit plaats en wat zijn de voor en nadelen hiervan en hoe wordt dit in de praktijk bij bedrijven uitgevoerd. Waar verwijnt SBC en waar niet? Er zal nagegaan worden wat de voor en nadelen zijn van SBC en of men terug zal keren naar smart endpoints en waarom. Het onderzoek zal uitgevoerd worden in het bedrijf Orbid, waar zal nagegaan worden wat de huidige structuur is van het bedrijf en wat hun huidige oplossingen zijn naar de klanten toe nu en in de toekomst. Ook zal gekeken worden of we terug zullen gaan naar decentrale opslag in plaats van centraal en wat de evolutie is hiervan. De verwachting zal zijn dat SBC stilaan zal overschakelen naar alternatieven zoals VDI en eventueel terug zal keren naar smart end devices maar dit nooit helemaal zal verdwijnen. De conclusie zal zijn dat bedrijven altijd zullen zoeken naar de beste en goedkoopste oplossigngen en altijd zullen meegaan met hun tijd.
}

%---------- Onderzoeksdomein en sleutelwoorden --------------------------------
% TODO: Sleutelwoorden:
%
% Het eerste sleutelwoord beschrijft het onderzoeksdomein. Je kan kiezen uit
% deze lijst:
%
% - Mobiele applicatieontwikkeling
% - Webapplicatieontwikkeling
% - Applicatieontwikkeling (andere)
% - Systeembeheer
% - Netwerkbeheer
% - Mainframe
% - E-business
% - Databanken en big data
% - Machineleertechnieken en kunstmatige intelligentie
% - Andere (specifieer)
%
% De andere sleutelwoorden zijn vrij te kiezen

\Keywords{Onderzoeksdomein. Citrix --- RDS --- Orbid} % Keywords
\newcommand{\keywordname}{Sleutelwoorden} % Defines the keywords heading name

%---------- Titel, inhoud -----------------------------------------------------

\begin{document}

\flushbottom % Makes all text pages the same height
\maketitle % Print the title and abstract box
\tableofcontents % Print the contents section
\thispagestyle{empty} % Removes page numbering from the first page

%------------------------------------------------------------------------------
% Hoofdtekst
%------------------------------------------------------------------------------

% De hoofdtekst van het voorstel zit in een apart bestand, zodat het makkelijk
% kan opgenomen worden in de bijlagen van de bachelorproef zelf.
%---------- Inleiding ---------------------------------------------------------

\section{Introductie} % The \section*{} command stops section numbering
\label{sec:introductie}
Onderzoek naar Server based computing, hoe ziet de toekomst eruit voor SBC ? verdwijnt RDS/Citrix in de komende 5 jaar ? Deze onderzoeksvraag komt van een medewerker van het bedrijf Orbid.
\begin{itemize}
  \item Meer en meer bedrijven willen af van SBC en zullen overschakelen naar alternatieven.
  \item Maar wat zijn de alternatieven en wat zijn de voor-en nadelen van SBC? Zullen Citrix en Microsoft RDS nog altijd even toepasbaar zijn in de nabije toekomst.
  \item Hoe zal de opslag van data in toekomst gebeuren en met welke programma's zal men werken. Gaat men terug naar centrale opslag in plaats van decentraal en wat is de historiek en toekomst hiervan.
\end{itemize}

%---------- Stand van zaken ---------------------------------------------------

\section{State-of-the-art}
\label{sec:state-of-the-art}

Er zijn nog geen gelijkaardige onderzoeken gebeurd aangezien dit een redelijk nieuw onderwerp is. De meeste bedrijven maken nog gebruik van server based computing via programma's zoals Citrix of RDS. Hoewel Citrix niet altijd de goedkoopste en makkelijkste oplossing is wil men naar de toekomst toe zoeken naar eventuele makkelijker oplossingen zoals VDI (Citrix XenDesktop)~\autocite{GB2014} . Er zijn wel al verschillende artikels verschenen waarbij men kijkt wat de toekomst is van SBC. Ook wat de toekomst is voor Citrix/RDS en hoe men meer en meer overschakelt naar VDI ~\autocite{Madden2017}. Maar echt onderzoeken of wetenschappelijke artikels vindt men niet snel weer.


% Voor literatuurverwijzingen zijn er twee belangrijke commando's:
% \autocite{KEY} => (Auteur, jaartal) Gebruik dit als de naam van de auteur
%   geen onderdeel is van de zin.
% \textcite{KEY} => Auteur (jaartal)  Gebruik dit als de auteursnaam wel een
%   functie heeft in de zin (bv. ``Uit onderzoek door Doll & Hill (1954) bleek
%   ...'')

%---------- Methodologie ------------------------------------------------------
\section{Methodologie}
\label{sec:methodologie}

Voor het onderzoek uit te voeren zal eerst een grondige literatuurstudie moeten uitgevoerd worden. Hierin zal zoveel mogelijk informatie worden verzameld omtrent de het te onderzoeken onderwerp. Nagaan wat de toekomst is van SBC, wat de alternatieven zijn, wat de bedrijven vandaag de dag allemaal gebruiken enzovoort.

Daarna zal nagegaan worden bij het bedrijf Orbid wat hun huidige structuur is en wat hun zicht is naar de toekomst toe omtrent SBC. Gaan ze hun plan van aanpak veranderen naar de klanten toe of blijft deze hetzelfde en welke software/hardware zijn ze van plan te gebruiken naar de toekomst toe. Er zal ook nagegaan worden hoe de huidige markt vandaag in elkaar zit en wat de vereisten zijn van de klanten.

%---------- Verwachte resultaten ----------------------------------------------
\section{Verwachte resultaten}
\label{sec:verwachte_resultaten}

De verwachte resultaten van dit onderzoek zal zijn dat Orbid nog altijd voornamelijk gebruik zal maken van SBC omdat dit nog steeds de goedkopere oplossing is. Maar naar de toekomst toe meer geneigd zullen zijn om te kiezen voor virtual desktop infrastructure oplossingen. Men zal nog vaak kiezen voor het gebruik van Citrix (zowel XenApp als XenDesktop) als RDS, maar dit zal naar de toekomst toe veranderen. De opslag van data zal ook nog steeds centraal gebeuren.

%---------- Verwachte conclusies ----------------------------------------------
\section{Verwachte conclusies}
\label{sec:verwachte_conclusies}

De verwachte conclusie van dit onderzoek zal zijn dat de men meer en meer zal overschakelen naar VDI oplossingen in plaats van SBC. Data zal nog steeds centraal opgeslaan worden maar naar de toekomst toe kan dit veranderen. Het gebruik van Microsoft RDS zal sterk dalen en voor Citrix zal men meer geneigd zijn om te kiezen voor XenDesktop in plaats van XenApp, dat meer VDI gericht is.



%------------------------------------------------------------------------------
% Referentielijst
%------------------------------------------------------------------------------
% TODO: de gerefereerde werken moeten in BibTeX-bestand ``voorstel.bib''
% voorkomen. Gebruik JabRef om je bibliografie bij te houden en vergeet niet
% om compatibiliteit met Biber/BibLaTeX aan te zetten (File > Switch to
% BibLaTeX mode)

\phantomsection
\printbibliography[heading=bibintoc]

\end{document}
